\section{Wyniki pracy}

%------------------------------------------------

\begin{frame}
    \frametitle{Uruchomienie symulatorów}
    \begin{figure}
        \includegraphics[scale=0.20]{./images/velma+omnivelma_cropped.png}
        \caption{Modele korpusu i bazy mobilnej w jednym świecie symulacji}
    \end{figure}
\end{frame}

%------------------------------------------------

\begin{frame}
    \frametitle{Unifikacja modeli}
    \only<1>{\begin{figure}
        \includegraphics[scale=0.27]{./images/omnivelmomobil-cropped.png}
        \caption{Wstępnie połączone modele korpusu i bazy mobilnej (bez czujników)}
    \end{figure}} 

    \only<2>{\begin{figure}
        \includegraphics[scale=0.19]{./images/omnivelmobil-final-cropped.png}
        \caption{Wstępnie połączone modele korpusu i bazy mobilnej}
    \end{figure}} 

    \only<3>{\begin{figure}
        \includegraphics[scale=0.22]{./images/working_lidars_cropped.png}
        \caption{Połączone modele z uruchomionymi czujnikami}
    \end{figure}} 
\end{frame}

%------------------------------------------------

\begin{frame}
    \frametitle{Proste zadanie manipulacji}
    \begin{figure}
        \includegraphics[scale=0.35]{./images/velma_stero_task.png}
        \caption{Zadanie przenoszenia puszki ze stolika na stolik}
    \end{figure}
\end{frame}

%------------------------------------------------

\begin{frame}
	\frametitle{Projekt koncepcyjny systemu}
	\bigskip
	\begin{figure}
        \includegraphics[scale=0.5]{./images/example_bdd.png}
        \caption{Poglądowa struktura systemu w SySML}
    \end{figure}
\end{frame}

%------------------------------------------------