\section{Podsumowanie}

\begin{frame}
    \frametitle{Podsumowanie}
    W ramach pracy zrealizowane zostały następujące zadania:
    \begin{itemize}
        \item konwersja modelu bazy mobilnej do formatu URDF,
        \item integracja modeli bazy i części manipulacyjnej,
        \item przygotowanie nowego sterownika bazy mobilnej,
        \item stworzenie planera ruchu,
        \item wykonanie testu funkcjonalnego,
        \item testy sterownika ruchu bazy.
    \end{itemize}
\end{frame}

\begin{frame}[plain]
    \addtocounter{framenumber}{-1}
    \frametitle{Bibliografia}
    \footnotesize{
    \begin{thebibliography}{99} % Beamer does not support BibTeX so references must be inserted manually as below
    \bibitem[1]{docsVelma} D. Seredyński
    \newblock Documentation of control system of WUT Velma Robot, 2017.
    \newblock \url{https://rcprg-ros-pkg.github.io/velma\_docs/}.
    
    \bibitem[2]{walas} P. Walas
    \newblock Nawigacja robota mobilnego w bezpośrednim otoczeniu człowieka
    \newblock \emph{Praca dyplomowa inżynierska}, Promotor: dr hab. inż. Wojciech Szynkiewicz, 2018.
    
    \bibitem[3]{bezpieczenstwo} T.Winiarski
    \newblock Wybrane aspekty bezpieczeństwa w badaniach robotów usługowych
    \newblock \emph{Politechnika Warszawska}, materiały do Sterowania i Symulacji Robotów, 2018.
    
    \bibitem[4]{sterowanie} T. Winiarski, C. Zieliński
    \newblock Podstawy sterowania siłowego robotów
    \newblock{Pomiary Automatyka Robotyka}, 6/2008, strony 5-10.
    
    \end{thebibliography}
    }
    \end{frame}

\begin{frame}[plain]
\addtocounter{framenumber}{-1}
\Huge{\centerline{Dziękuję za uwagę}}
\end{frame}
